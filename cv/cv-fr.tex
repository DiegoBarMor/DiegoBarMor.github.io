\name{Diego Barquero Morera}
\address{
    Courriel: \href{mailto:diego.barquero-morera@phys.ens.fr}{diego.barquero-morera@phys.ens.fr}
    $\qquad$ \href{https://diegobarmor.github.io/}{Site web}
    $\qquad$ \href{https://github.com/DiegoBarMor}{GitHub}
    $\qquad$ \href{https://www.linkedin.com/in/diego-barquero-morera-98863020b/}{LinkedIn}
}
\address{Portable: (+33) 745 105 417 $\qquad$ Adresse: Île-de-France, Créteil, 5 Rue des Bordières}
\begin{document}


%--------------------------------------------------------------------
%    EXPERIENCE SECTION
%--------------------------------------------------------------------
\begin{rSection}{Expérience}
    \vspace{\baselineskip}

    \begin{rSubsectionShort}{
        \bf Tuteur en mathématiques @ \href{https://u-paris.fr/en/}{Université Paris Cité (UPC)}
    }{07.2024-03.2027}{Cours pour les étudiants de Master et développement du \href{https://diegobarmor.github.io/maths-tutorship/}{site web complémentaire}.}{Paris, France}
    \end{rSubsectionShort}

    \begin{rSubsectionShort}{
        \bf Doctorant @ \href{https://www.lpens.ens.psl.eu/}{École Normale Supérieure (ENS)}
    }{03.2024-03.2027}{Optimisation d'un champ de force coarse-grained pour l'ARN. Analyse de données, \\ machine learning avec Python, PyTorch, Bash. Programmation en Fortran, C++, C.}{Paris, France}
    \end{rSubsectionShort}

    \begin{rSubsectionShort}{
        \bf Stage de Master @ \href{https://u-paris.fr/en/}{Université Paris Cité (UPC)}
    }{03.2023-02-2024}{Caractérisation des sites de liaison t leur visualisation avec Unity/C\#.}{Paris, France}
    \end{rSubsectionShort}

    \begin{rSubsectionShort}{
        \bf Tuteur en chimie organique @ \href{https://www.tec.ac.cr/}{Costa Rica Institute of Technology (ITCR)}
    }{07.2017-11.2018}{Cours de chimie organique pour les étudiants de Licence.}{Cartago, Costa Rica}
    \end{rSubsectionShort}

\end{rSection}


%--------------------------------------------------------------------
%    EDUCATION SECTION
%--------------------------------------------------------------------
\begin{rSection}{Formation}
    \vspace{\baselineskip}

    \begin{rSubsectionShort}{
        \bf PhD en biophysique computationnelle @ \href{https://www.lpens.ens.psl.eu/}{École Normale Supérieure (ENS)}
    }{03.2024-03.2027}{Optimisation d'un champ de force coarse-grained pour l'ARN.}{Paris, France}
    \end{rSubsectionShort}

    \begin{rSubsectionShort}{
        \bf Master en biologie quantitative et computationnelle \\\@ \href{https://www.unitn.it/}{Università degli Studi di Trento (Unitn)}
    }{09.2021-12.2023}{Caractérisation et visualisation des sites de liaison. \\30L/30 avec distinction.}{Trento, Italy}
    \end{rSubsectionShort}

    \begin{rSubsectionShort}{
        \bf Programme d'échange @ \href{https://www.tum.de/en/}{Technical University of Munich (TUM)}
    }{04.2019-09.2019}{Cours de chimie organique et de biologie moléculaire.}{Munich, Germany}
    \end{rSubsectionShort}

    \begin{rSubsectionShort}{
        \bf Licence en génie biotechnologique \\\@ \href{https://www.tec.ac.cr/}{Costa Rica Institute of Technology (TEC)}
    }{02.2016-09.2021}{Développement d'algorithmes pour la comparaison des flux métaboliques. \\100/100 avec distinction.}{Cartago, Costa Rica}
    \end{rSubsectionShort}

\end{rSection}


%--------------------------------------------------------------------
%    COMPETENCES SECTION
%--------------------------------------------------------------------
\begin{rSection}{Compétences}
    \vspace{\baselineskip}
    \subsection*{Expérience avec les logiciels :}
    \begin{itemize}
        \item \textbf{Niveau élevé :} Python, Bash. Linux, VS Code, Git, MS Office.
        \item \textbf{Niveau intermédiaire :} PyTorch, LaTeX, Markdown, HTML, Javascript, CSS, \\\ Fortran, C, C\#, C++, Java, Kotlin. Unity3D, Android Studio, Visual Studio.
        \item \textbf{Notions :} Batch, R, MATLAB, Tcl, Rust, Lua, GLSL.
    \end{itemize}

    \subsection*{Langues :}
    \begin{itemize}
        \item \textbf{Courant :} Espagnol (natif), Anglais (C1.2), Italien (B2/C1).
        \item \textbf{Notions :} Allemand (A2/B1), Français (A2).
    \end{itemize}
\end{rSection}

% \vspace{\baselineskip}
\begin{flushright} \small{Mis à jour le : \today} \end{flushright}

\end{document}
